\documentclass[12pt]{scrartcl}

\usepackage[sexy]{evan}
\usepackage{parskip}

\title{Class Notes}
\date{June 18}

\begin{document}

\maketitle

\pagebreak

\section{Description}

Continued look at vectors.

\section{Notes}

Aside from the dot product, there is one more product to consider between vectors. This, of course, is the cross product, an operation that results in a vector output and does not follow the typical rules of multiplication that the cross product follows. For example, $\vec{a} \times \vec{b} \neq \vec{b} \times \vec{a}$ in most cases. In fact, $\vec{a} \times \vec{b} = -(\vec{b} \times \vec{a})$.

\begin{definition}[Vector Cross Product]
    In $\RR_{3}$ space, the cross product of $\vec{u} = \left<a_{1}, a_{2}, a_{3}\right>$ and $\vec{v} = \left<b_{1}, b_{2}, b_{3}\right>$ is
    \[\vec{u} \times \vec{v} = \det \begin{bmatrix} \hat{\imath} & \hat{\jmath} & \hat{k} \\ a_{1} & a_{2} & a_{3} \\ b_{1} & b_{2} & b_{3} \end{bmatrix}.\]
\end{definition}

\begin{remark}
    Take note that the vector cross product is only generally defined in three dimensions (and seven, but that's beyond the scope of this course).
\end{remark}

\begin{example}
    Find the cross product of $\vec{u} = \left<4, 3, 1\right>$ and $\vec{v} = \left<5, -7, 2\right>$.

    \begin{soln}
        Using our above definition, we see that
        \begin{align*}
            \vec{u} \times \vec{v} &= \det \begin{bmatrix}
                \hat{\imath} & \hat{\jmath} & \hat{k} \\
                4 & 3 & 1 \\
                5 & -7 & 2
            \end{bmatrix} \\
            &= \left<3 \cdot 2 - (-7) \cdot 1, -4 \cdot 2 + 5 \cdot 1, 4 \cdot (-7) - 5 \cdot 3\right> \\
            &= \left<13, -3, -43\right>.
        \end{align*}
    \end{soln}
\end{example}

\begin{remark}
    Another important note is that the cross product always returns a vector that is perpendicular to both input vectors (unless they are parallel, in which the zero-vector is returned).
\end{remark}

\begin{exercise}
    Prove that the cross product from the example above is perpendicular to both input vectors.
\end{exercise}

\begin{exercise}
    Prove that in general, the cross product returns a perpendicular vector to both input vectors.
\end{exercise}

Going back to the above definition of a cross product, we can arrive at a trigonometric association for the cross product that is similar to the one established for the dot product.

\begin{proposition}
    Two vectors $\vec{u} = \left<a_{1}, a_{2}, a_{3}\right>$ and $\vec{v} = \left<b_{1}, b_{2}, b_{3}\right>$ satisfy
    \[\magnitude{\vec{u} \times \vec{v}} = \magnitude{\vec{u}} \magnitude{\vec{v}} \sin\theta\]
    where $\theta$ is the angle between $\vec{u}$ and $\vec{v}$.
\end{proposition}

In the beginning, I note that the cross product has slightly different multiplicative properties than the dot product. Because of this, I list the specific properties below.

\begin{itemize}
    \item \textbf{Anticommutivity:} $\vec{a} \times \vec{b} = -(\vec{a} \times \vec{b})$.
    \item \textbf{Scalar Multiplication:} $(k\vec{a}) \times \vec{b} = \vec{a} \times (k\vec{b}) = k(\vec{a} \times \vec{b})$.
    \item \textbf{Distributivity:} $\vec{a} \times (\vec{b} + \vec{c}) = \vec{a} \times \vec{b} + \vec{a} \times \vec{c}$.
    \item \textbf{Non-Associativity:} In general, $(\vec{a} \times \vec{b}) \times \vec{c} \neq \vec{a} \times (\vec{b} \times \vec{c})$.
\end{itemize}

Now, we explore an example that utilizes the trigonometric representation above. As we go through, we must keep in mind the properties above.

\begin{example}[Volume of a Parallelepiped]
    Prove that the volume of a parallelepiped is
    \[V = \vec{u} \cdot (\vec{v} \times \vec{w})\]
    where the vectors represent the three sets of parallel sides.

    \begin{proof}
        Rearranging the above we have
        \begin{align*}
            V &= \magnitude{\vec{u}} \magnitude{\vec{v} \times \vec{w}} \cos\theta \\
            &= \magnitude{\vec{u}} \magnitude{\vec{v}} \magnitude{\vec{w}} \cos\theta \sin\alpha
        \end{align*}
        where $\theta$ is the angle between $\vec{u}$ and the altitude $h$ between the bases determined by sides in the directions of $\vec{v}$ and $\vec{w}$ and $\alpha$ is the angle between $\vec{v}$ and $\vec{w}$. It is easy to see that $h = \magnitude{\vec{u}}\cos\theta$ and the area of the base $b = \magnitude{\vec{v}} \magnitude{\vec{w}} \sin\alpha$. So we attain $V = bh$, the standard volume formula.
    \end{proof}
\end{example}

Lines in 3-dimensional space are not written in the same way that they are in two-dimensional space. In fact, they must be paramaterized.

\begin{definition}
    A line in $\RR_{3}$ space is written in the form $\vec{p} + t\vec{u}$ where $\vec{p}$ represents a position vector, $\vec{u}$ represents a direction vector, and $t$ is a varied parameter.
\end{definition}

\begin{remark}
    Often, $\vec{u}$ in the above is a unit vector, but this is not necessary.
\end{remark}



\end{document}