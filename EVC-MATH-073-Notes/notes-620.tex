\documentclass[11pt]{scrartcl}

\usepackage[sexy]{evan}
\usepackage{parskip}

\begin{document}

\section{June 20, 2024}

Lines are always defined using a point and a direction. For example, standard $\RR_{2}$ lines are written in the form $y = mx + b$, where $(0, b)$ is a point on the line and $m$ represents the slope, or direction. However, lines in $\RR_{3}$ space and higher spaces cannot be defined with one simple equation only containing scalar inputs. In the last class, we gave the definition of a line $\ell : \vec{p} + t\vec{u}$. However, lines can also be written as a collection of parametric equations as seen in the example below.

\begin{example}
    Line $\ell$ contains point $\left<1, 1, 2\right>$ and has direction $\left<1, 2, 4\right>$. Write a collection of parametric equations that represent $\ell$.

    \begin{soln}
        To write each parametric equation, we take the corresponding components of a known point and the direction. For example, when solving for $x$, we have
        \[x = p_{x} + u_{x}t\]
        Doing this for each variable yields an answer of
        \[x = 1 + t\]
        \[y = 1 + 2t\]
        \[z = 2 + 4t.\]
    \end{soln}
\end{example}

If we solve for $t$ in each of the three equations above, we arrive at a symmetric equation of a line.

\begin{corollary}[Symmetric Equation of a Line]
    Given a line $\ell$ containing point $\vec{p}$ with direction $\vec{u}$, $\ell$ can be represented by the symmetric equation
    \[\frac{x - p_{x}}{u_{x}} = \frac{y - p_{y}}{u_{y}} = \frac{z - p_{z}}{u_{z}}.\]
\end{corollary}

Now, if you recall $\RR_{2}$ space, lines are either parallel or intersecting. However, in $\RR_{3}$ space and beyond, lines can also be skew.

\begin{definition}[Skew Lines]
    Two lines in space are skew if they are neither parallel nor intersecting.
\end{definition}

For the next two propositions consider distinct lines $\ell_{1} : \vec{p}_{1} + t\vec{u}_{1}$ and $\ell_{2} : \vec{p}_{2} + t\vec{u}_{2}$.

\begin{theorem}[Parallel Lines]
    Lines $\ell_{1}$ and $\ell_{2}$ are parallel if and only if $\vec{u}_{1} = k\vec{u}_{2}$ for some scalar constant $k$.
\end{theorem}

\begin{theorem}[Intersecting Lines]
    Lines $\ell_{1}$ and $\ell_{2}$ intersect if and only if there exists $t$ and $s$ such that
    \[\vec{p}_{1} + t\vec{u}_{1} = \vec{p}_{2} + s\vec{u}_{2}.\]
    Otherwise, the lines are parallel or skew.
\end{theorem}

\begin{example}
    Determine if the lines $\ell_{1} : \frac{x}{1} = \frac{y - 1}{-1} = \frac{z - 2}{3}$ and $\ell_{2} : \frac{x - 2}{2} = \frac{y -  3}{2} = \frac{z}{7}$ are parallel, intersecting, or skew.

    \begin{soln}
        It's clear to see that the direction vectors are not multiples of each other, so the lines are not parallel. We now attempt to find a common intersection point. First, we look at $x$. We know that the $x$-valeus are the same when
        \[t = 2s + 2.\]
        Now, looking at $y$ and doing the same, we see that
        \[1 - t = 2s + 3.\]
        We add the two equations and see that
        \[1 = 4s + 5 \implies s = -1, t = 0.\]
        So, the lines are skew.
    \end{soln}
\end{example}

Now, we move to planes. In previous notes, we deduced the equation of a plane, but let's prove why.

\begin{example}[Equation of a Plane]
    Find, with proof, the equation of a plane in $\RR_{3}$.

    \begin{proof}
        Let plane $\mathcal{P}$ have a normal vector $\vec{u} = \left<a, b, c\right>$. Also, define a vector $\vec{v} = \left<x - x_{0}, y - y_{0}, z - z_{0}\right>$ be a vector from a fixed point in $\mathcal{P}$ to any arbitrary point in $\mathcal{P}$. Since $\vec{u}$ is a normal vector to $\mathcal{P}$, we know that $\vec{u} \perp \vec{v}$. So,
        \[\vec{u} \cdot \vec{v} = a(x - x_{0}) + b(y - y_{0}) + c(z - z_{0}) = 0.\]
        Since $\vec{v}$ represents a vector to any arbitrary point in $\mathcal{P}$, we have the equation of the plane.
    \end{proof}
\end{example}

\begin{example}
    Find the equation of the plane containing $X = (2, 3, 5)$, $Y = (4, -1, 6)$, and $Z = (1, 9, -7)$.

    \begin{soln}
        We have $\vec{XY} = \left<2, -4, 1\right>$ and $\vec{YZ} = \left<-3, 10, -13\right>$. We already know a fixed point on our plane, so we are left to find a normal vector. To do so, we take the cross product of the two vectors determined above.
        \begin{align*}
            \vec{XY} \times \vec{YZ} &= \det\begin{bmatrix}
                \hat{\imath} & \hat{\jmath} & \hat{k} \\
                2 & -4 & 1 \\
                -3 & 10 & -13
            \end{bmatrix} \\
            &= \left<42, 23, 8\right>.
        \end{align*}
        Without loss of generality, we select $X$ as a point within the plane and have the equation
        \[42(x - 2) + 23(y - 3) + 8(z - 5) = 0.\]
    \end{soln}
\end{example}

\begin{example}
    Determine if line $\ell : \left<1 + t, 2 - t, 4 - 3t\right>$ and plane $\mathcal{P} : 5x + 2y + z = 1$ are parallel.

    \begin{soln}
        We know that $\ell \parallel \mathcal{P}$ if and only if the direction vector of $\ell$ is orthogonal to a normal vector of $\mathcal{P}$. One such normal vector is $\vec{u} = \left<5, 2, 1\right>$. The direction of $\ell$ is $\vec{v} = \left<1, -1, -3\right>$. So,
        \[\vec{u} \cdot \vec{v} = 5 - 2 - 3 = 0.\]
        We thus conclude that $\ell \parallel \mathcal{P}$.
    \end{soln}
\end{example}

\end{document}