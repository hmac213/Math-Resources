\documentclass[11pt]{scrartcl}

\usepackage[sexy]{evan}
\usepackage[parfill]{parskip}
\usepackage{amsmath}
\usepackage{amssymb}
\usepackage{graphicx}

\graphicspath{ {./Images/} }

\newcommand*{\problemfont}{\sffamily\bfseries}

\title{Extra Problems}
\author{Henry McNamara}

\begin{document}

\begin{problem}
    Some number of dots are chosen from the vertices of a $3 \times 3$ grid so that at least one dot is chosen. In how many ways can lines be drawn sequentially so that they satisfy the conditions of those video games?
\end{problem}

\vspace{-\baselineskip}\rule{\textwidth}{0.4pt}

\pagebreak

\begin{problem}
    Jen selects a random two-digit integer to place on a lottery ticket. A string of 100 random digits is generated and if Jen's number appears in any base $b$ between 2 and 10 at some point in the sequence, then she wins. What is the probability she wins?
\end{problem}

\vspace{-\baselineskip}\rule{\textwidth}{0.4pt}

\pagebreak

\begin{problem}
    Find the least integer $R$ such that there exist at least 10 unique triangles with integer sidelengths whose circumradius is equal to $R$.
\end{problem}

\vspace{-\baselineskip}\rule{\textwidth}{0.4pt}

\pagebreak

\begin{problem}
    Triangle $ABC$ has sidelengths $AB = 4$, $BC = 3$, and $AC = 5$. The $A$-excircle of $ABC$ intersects the circumcircle of $ABC$ at points $M$ and $N$. Find $MN$.
\end{problem}

\vspace{-\baselineskip}\rule{\textwidth}{0.4pt}

\begin{center}
    \begin{asy}
        size(8cm);

        pair A,B,C,O,D,M,N,P1,P2;
        B = origin;
        C = (3,0);
        A = (0,4);
        O = (1.5,2);
        D = (2,-2);
        path c1 = circle(A,B,C);
        path c2 = circle(D,2);
        pair[] MN;
        MN = intersectionpoints(circle(A,B,C),circle(D,2));
        real s1, s2, s3, s4;
        s1 = 0;
        s2 = 1;
        s3 = 3/5;
        s4 = 4/5;
        P1 = (2*s1,-2*s2);
        P2 = C + (s3,-s4);

        draw(A -- B -- C -- cycle, RGB(63,159,255));
        draw(c1, RGB(63,159,255));
        draw(c2, RGB(63,159,255));
        draw(MN[0] -- MN[1], red);
        draw(B -- P1, dashed+RGB(63,159,255));
        draw(C -- P2, dashed+RGB(63,159,255));

        dot("$A$",A,dir(90));
        dot("$B$",B,dir(180));
        dot("$C$",C,dir(0));
        dot("$O$",O,dir(90));
        dot("$I_{A}$",D,S);
        dot("$M$",MN[0],W);
        dot("$N$",MN[1],E);
        dot("$P$",P1,W);
        dot("$Q$",P2,E);
    \end{asy}
\end{center}

We put the triangle and circles in the Cartesian plane with $A = (5, 12)$, $B = (0, 0)$, and $C = (14, 0)$. We proceed to find $O = \left(7, \frac{33}{8}\right)$ by finding the intersection of the perpendicular bisectors of $BC$ and $AB$. We find the circumradius with
\[R = \frac{AB \cdot AC \cdot BC}{4[ABC]},\]
which gives an answer of $\frac{65}{8}$. So, the equation of $\omega$ is
\[(x - 7)^{2} + (y - \frac{33}{8})^{2} = \left(\frac{65}{8}\right)^{2}\]
To find $I_{A}$, we use the formula
\[r_{A} = \frac{[ABC]}{s - BC}\]
where $s$ is the semiperimeter of $ABC$. So, $r_{A} = 12$, which makes the $y$-coordinate of $I_{A}$ also equal to $-12$. To find the $x$-coordinate, notice that $AP = AQ$. Let $BP = m$ and $CQ = n$. We now have
\[13 + m = 15 + n\]
and $m + n = 14$. This implies that $m = 8$, which makes $I_{A} = (-12,8)$. The equation for $\omega_{A}$ is then
\[(x - 8)^{2} + (y + 12)^{2} = 144.\]
Equating the two circles yields a new answer when we change the triangle to a 3-4-5


\pagebreak

\begin{problem}
    John picks a random integer between 1 and 1000 inclusive. Jen attempts to guess the number. Each turn, she says an integer and John will either tell her that his number is higher, lower, or that she got it correct. Assuming Jen guesses optimally, find the expected number of guesses until she correctly identifies John's number.
\end{problem}

\vspace{-\baselineskip}\rule{\textwidth}{0.4pt}

\pagebreak

\end{document}