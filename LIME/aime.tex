\documentclass[11pt]{scrartcl}

\usepackage[sexy]{evan}
\usepackage[parfill]{parskip}
\usepackage{amsmath}
\usepackage{amssymb}
\usepackage{graphicx}

\graphicspath{ {./Images/} }

\title{2024 AIME I Solutions}
\author{Henry McNamara}

\begin{document}

\begin{problem}
    Every morning Aya goes for a 9-kilometer-long walk and stops at a coffee shop afterwards. When she walks at a constant speed of $s$ kilometers per hour, the walk takes her 4 hours, including $t$ minutes spent in the coffee shop. When she walks at $s + 2$ kilometers per hour, the walk takes her 2 hours and 24 inutes, including $t$ minutes spent in the coffee shop. Suppose Aya walks at $s + \frac{1}{2}$ kilometers per hour. Find the number of minutes the walk takes her, including the $t$ minutes spent in the coffee shop.
\end{problem}

\rule{\textwidth}{0.4pt}

Using the formula for distance,
\[d = rt\]
we get the following system of equations:
\[9 = s(4 - a)\]
\[9 = (s + 2)(\frac{12}{5} - a)\]
where $a$ is the time spent in the coffee shop, expressed in hours. Equation the two, we get
\[4s - as = \frac{12}{5}s - as + \frac{24}{5} - 2a\]
\[s = 3 - \frac{5}{4}a\]
which we can substitute to find
\[9 = (3 - \frac{5}{4}a)(4 - a)\]
\[9 = 12 - 8a + \frac{5}{4}a^{2}\]
\[a = \frac{2}{5} \implies s = \frac{5}{2}.\]
Now, let $h$ be the total time taken for the final walk, yielding
\[9 = 3(h - \frac{2}{5})\]
\[h = \frac{17}{5}.\]
This equates to a total time of $\boxed{204}$ minutes.

\pagebreak

\begin{problem}
    There exist real numbers $x$ and $y$, both greater than 1, such that $\log_{x}\left(y^{x}\right) = \log_{y}\left(x^{4y}\right) = 10$. Find $xy$.
\end{problem}

\rule{\textwidth}{0.4pt}

Multiplying the two logarithmic equations yields
\[\log_{x}\left(y^{x}\right)\log_{y}\left(x^{4y}\right) = 100\]
\[4xy\log_{x}y\log_{y}x = 100.\]
Since $\log_{a}b = \frac{1}{\log_{b}a}$,
\[4xy = 100 \implies xy = \boxed{25}.\]

\pagebreak

\begin{problem}
    Alice and Bob play the following game. A stack of $n$ tokens lies before them. The players take turns with Alice going first. On each turn, the player removes either 1 token or 4 tokens from the stack. Whoever removes the last token wins. Find the number of positive integers $n$ less than or equal to 2024 for which there exists a strategy for Bob that guarentees that Bob will win the game regardless of Alice's play.
\end{problem}

\rule{\textwidth}{0.4pt}

\pagebreak

\begin{problem}
    Jen enters a lottery by selecting four distinct elements of $S = \{1, 2, 3, 4, 5, 6, 7, 8, 9, 10\}$. Then four distinct elements of $S$ are drawn at random. Jen wins a prize if at least two of her numbers are drawn, and she wins the grand prize if all four of her numbers are drawn. The probability that Jen wins the grand prize given that Jen wins a prize is $\frac{m}{n}$, where $m$ and $n$ are relatively prime positve integers. Find $m + n$.
\end{problem}

\rule{\textwidth}{0.4pt}

When choosing $k$ elements to match, there are ${4 \choose k}$ ways to choose the $k$ elements from Jen's numbers. There are then ${6 \choose 4 - k}$ remaining numbers to choose. So, our desired probability is
\[\frac{{4 \choose 4}{6 \choose 0}}{{4 \choose 2}{6 \choose 2} + {4 \choose 3}{6 \choose 1} + {4 \choose 4}{6 \choose 0}} = \frac{1}{115}.\]
The desired sum is then $1 + 115 = \boxed{116}$.

\pagebreak

\begin{problem}
    Rectangle $ABCD$ has dimensions $AB = 107$ and $BC = 16$, and rectangle $EFGH$ has dimensions $EF = 184$ and $FG = 17$. Points $D$, $E$, $C$, and $F$ lie on line $DF$ in that order, and $A$ and $H$ lie on opposite sides of line $DF$, as shown. Points $A$, $D$, $H$, and $G$ lie on a common circle. Find $CE$.
\end{problem}

\rule{\textwidth}{0.4pt}

\pagebreak

\begin{problem}
    Consider paths of length 16 that follows the lines from the lower left corner to the upper right corner on an $8 \times 8$ grid. Find the number of such paths that change direction exactly four times, as in the examples shown below.
\end{problem}

\rule{\textwidth}{0.4pt}

We consider grids with the first move going to the right and then multiply by two to account for the first move going upward. The four direction changes implies that there are three rightward paths of non-zero length and two upward paths of non-zero length. From stars and bars, we see that there are a total of
\[{7 \choose 5}{7 \choose 1} = 21 \cdot 7 = 147\]
valid paths. Accounting for the first move being vertical, we get an answer of $\boxed{294}$.

\pagebreak

\begin{problem}
    Find the greatest possible real part of
    \[(75 + 117i)z + \frac{96 + 144i}{z},\]
    where $z$ is a complex number with $\abs{z} = 4$. Here $i = \sqrt{-1}$.
\end{problem}

\rule{\textwidth}{0.4pt}

To make computation easier, we eliminate the fraction in the expression:
\begin{align*}
    (75 + 117i)z + \frac{96 + 144i}{z} &= (75 + 117i)z + \frac{(96 + 144i)\overline{z}}{\abs{z}^{2}} \\
    &= (75 + 117i)z + (6 + 9i)\overline{z}
\end{align*}
If we let $z = a + bi$, we get
\begin{align*}
    (75 + 117i)z + \frac{96 + 144i}{z} &= (75 + 117i)(a + bi) + (6 + 9i)(a - bi) \\
    &= 81a - 108b + 69bi + 126ai
\end{align*}
With the Cauchy-Schwarz inequality, we can maximize the real part of the above expression:
\[(3^{2} + (-4)^{2})(a^{2} + b^{2}) \geq (3a - 4b)^{2}\]
\[20 \geq 3a - 4b \implies 81a - 108b \leq \boxed{540}.\]

\pagebreak

\begin{problem}
    Eight circles of radius 34 can be placed tangent to side $\overline{BC}$ of $\triangle ABC$ so that the circles are sequentially tangent to each other, with the first circle being tangent to $\overline{AB}$ and the last circle being tangent to $\overline{AC}$, as shown. Similarly, 2024 circles of radius 1 can be placed tangent to $\overline{BC}$ in the same manner. The inradius of $\triangle ABC$ can be expressed as $\frac{m}{n}$, where $m$ and $n$ are relatively prime positive integers. Find $m + n$.
\end{problem}

\rule{\textwidth}{0.4pt}

\pagebreak

\begin{problem}
    Let $A$, $B$, $C$, and $D$ be points on the hyperbola $\frac{x^{2}}{20} - \frac{y^{2}}{24} = 1$ such that $ABCD$ is a rhombus whose diagonals intersect at the origin. Find the greatest real number that is less than $BD^{2}$ for all such rhombi.
\end{problem}

\rule{\textwidth}{0.4pt}

\pagebreak

\begin{problem}
    Let $\triangle ABC$ have side lengths $AB = 5$, $BC = 9$, and $CA = 10$. The tangents to the circumcircle of $\triangle ABC$ at $B$ and $C$ intersect at point $D$, and $\overline{AD}$ intersects the circumcircle at $P \neq A$. The length of $\overline{AP}$ is equal to $\frac{m}{n}$, where $m$ and $n$ are relatively prime positive integers. Find $m + n$.
\end{problem}

\rule{\textwidth}{0.4pt}

\pagebreak

\begin{problem}
    Each vertex of a regular octagon is independently colored either red or blue with equal probability. The probability that the octagon can then be rotated so that all of the blue vertices move to positions where there had been red vertices is $\frac{m}{n}$, where $m$ and $n$ are relatively prime positive integers. Find $m + n$.
\end{problem}

\rule{\textwidth}{0.4pt}

\pagebreak

\begin{problem}
    Define $f(x) = \abs{\abs{x} - \frac{1}{2}}$ and $g(x) = \abs{\abs{x} - \frac{1}{4}}$. Find the number of intersections of the graphs of
    \[y = 4g(f(\sin(2\pi x))) \;\;\; \text{and} \;\;\; x = 4g(f(\cos(3\pi y))).\]
\end{problem}

\rule{\textwidth}{0.4pt}

\pagebreak

\begin{problem}
    Let $p$ be the least prime number for which there exists an integer $n$ such that $n^{4} + 1$ is divisible by $p^{2}$. Find the least positive integer $m$ such that $m^{4} + 1$ is divisible by $p^{2}$.
\end{problem}

\rule{\textwidth}{0.4pt}

\pagebreak

\begin{problem}
    Let $ABCD$ be a tetrahedron such that $AB = CD = \sqrt{41}$, $AC = BD = \sqrt{80}$, and $BC = AD = \sqrt{89}$. There exists a point $I$ inside the tetrahedron such that the distances from $I$ to each of the faces of the tetrahedron are all equal. This distance can be written in the form $\frac{m\sqrt{n}}{p}$, where $m$, $n$, and $p$ are positive integers, $m$ and $p$ are relatively prime, and $n$ is not divisible by the square of any prime. Find $m + n + p$.
\end{problem}

\rule{\textwidth}{0.4pt}

\pagebreak

\begin{problem}
    Let $\mathcal{B}$ be the set of rectangular boxes with surface area 54 and volume 23. Let $r$ be the radius of the smallest sphere that can contain each of the rectangular boxes that are elements of $\mathcal{B}$. The value of $r^{2}$ can be written as $\frac{p}{q}$, where $p$ and $q$ are relatively prime positive integers. Find $p + q$.
\end{problem}

\rule{\textwidth}{0.4pt}

Let the side lengths of the rectangular boxes be $a$, $b$, and $c$. We have that $abc = 23$ and $ab + ac + bc = 27$. The radius of the sphere with minimum volume that encompasses any given rectangle is
\[r = \frac{\sqrt{a^{2} + b^{2} + c^{2}}}{2}.\]
From the first two conditions, we have that
\[\frac{1}{a} + \frac{1}{b} + \frac{1}{c} = \frac{27}{23}.\]
From Titu's inequality,
\[\frac{1}{a} + \frac{1}{b} + \frac{1}{c} \geq \frac{9}{a + b + c}\]
\[\frac{27}{23} \geq \frac{9}{a + b + c}\]
\[\frac{a + b + c}{9} \geq \frac{23}{27}\]
\[\frac{(a + b + c)^{2} - 2(ab + ac + bc)}{81} \geq \frac{529}{729} - \frac{2}{3}\]
\[\frac{a^{2} + b^{2} + c^{2}}{81} \geq \frac{43}{729}\]
\[\sqrt{a^{2} + b^{2} + c^{2}} \geq \frac{43}{81}\]
\[\frac{\sqrt{a^{2} + b^{2} + c^{2}}}{2} \geq \frac{43}{162}\]

\end{document}