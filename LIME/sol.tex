\documentclass[11pt]{scrartcl}

\usepackage[sexy]{evan}
\usepackage[parfill]{parskip}
\usepackage{amsmath}
\usepackage{amssymb}
\usepackage{graphicx}

\graphicspath{ {./Images/} }

\newcommand*{\problemfont}{\sffamily\bfseries}

\title{Late Invitational Math Exam Report}
\author{LIME Team}
\date{11 February 2024}

\begin{document}

\maketitle

\pagebreak

\section{Statistics}

\textbf{Number of Submissions:} 21 \\
\textbf{Mean:} 5.00 \\
\textbf{Median:} 4 \\
\textbf{Minimum:} 0 \\
\textbf{Maximum:} 12 \\
\textbf{Problem Solve Rates:} \\
\begin{center}
    \begin{asy}
        size(10cm);
        real x,y;
        real[] t;

        x = 1;
        y = 0.75;

        t[1] = 17;
        t[2] = 9;
        t[3] = 8;
        t[4] = 10;
        t[5] = 16;
        t[6] = 5;
        t[7] = 11;
        t[8] = 6;
        t[9] = 6;
        t[10] = 2;
        t[11] = 7;
        t[12] = 1;
        t[13] = 3;
        t[14] = 3;
        t[15] = 1;

        for(int i = 1; i < 16; ++i) {
            draw(((i - 1)*x,0) -- (i*x,0) -- (i*x,t[i]*y) -- ((i - 1)*x,t[i]*y) -- cycle);
            label(string(i),((i - 0.5)*x,0), S);
            label(string(t[i]),((i - 0.5)*x,t[i]*y), N);
        }
    \end{asy}
\end{center}

\pagebreak

\section{Analysis}

\subsection{Problem Analysis}

Here, we look at where we would place each problem if we were to rewrite the mock after the difficulty of a real AIME.

\begin{center}
    \begin{tabular}{r l | r}
        \# & Description & New Spot \\
        \hline
        1 & Palindromes & 1 \\
        2 & Seating Arrangements & 8 \\
        3 & Hexagon $ABCDEF$ & 7 \\
        4 & Expected Value Dice & 7 \\
        5 & Logs and Floors & 5 \\
        6 & Equal Area Triangles & 9 \\
        7 & Number Game & 7 \\
        8 & Perspective Parabola & 10 \\
        9 & Sequential Floors & 9 \\
        10 & Tetrahedron Surface Area & 13 \\
        11 & Sum of Powers & 11 \\
        12 & Alternating Subsequences & 15 \\
        13 & Fibonacci Sum & 12 \\
        14 & Polynomial Quadrilateral & 14 \\
        15 & Triangle Optimization & 15
    \end{tabular}
\end{center}

\subsection{Score Analysis}

Based on the difficulty of the problems, we believe that scores on the LIME translate to approximately the following AIME scores.

\begin{center}
    \begin{tabular}{l | r r r r r r r r r r r r r r r r}
        LIME Score & 0 & 1 & 2 & 3 & 4 & 5 & 6 & 7 & 8 & 9 & 10 & 11 & 12 & 13 & 14 & 15 \\
        \hline
        AIME Score & 0 & 3 & 5 & 6 & 7 & 8 & 9 & 9 & 9 & 10 & 11 & 12 & 13 & 14 & 15 & 15
    \end{tabular}
\end{center}

\pagebreak

\section{Solutions}

\subsection{Answer Key}

\begin{center}
    \begin{tabular}{r l | r}
        \# & Author & Answer \\
        \hline
        1 & Henry McNamara & 120 \\
        2 & Henry McNamara & 009 \\
        3 & Henry McNamara & 452 \\
        4 & Sarthak Jain & 626 \\
        5 & Sarthak Jain & 026 \\
        6 & Henry McNamara and Sarthak Jain & 019 \\
        7 & Henry McNamara & 429 \\
        8 & Henry McNamara & 017 \\
        9 & Henry McNamara and Sarthak Jain & 266 \\
        10 & Henry McNamara & 157 \\
        11 & Henry McNamara & 032 \\
        12 & Sarthak Jain & 500 \\
        13 & Sarthak Jain & 311 \\
        14 & Sarthak Jain & 021 \\
        15 & Henry McNamara & 100 \\
    \end{tabular}
\end{center}

\pagebreak

\subsection{Problem Solutions}

\begin{problem}
    A two-digit integer $\underline{a} \, \underline{b}$ is multiplied by 9. The resulting three-digit integer is of the form $\underline{a} \, \underline{c} \, \underline{b}$ for some digit $c$. Evaluate the sum of all possible $\underline{a} \, \underline{b}$.
\end{problem}

\vspace{-\baselineskip}\rule{\textwidth}{0.4pt}

{\problemfont Solution Author} Henry McNamara

Consider the equation
\[90a + 9b = 100a + 10c + b\]
\[4b = 5a + 5c.\]
It follows that $b = 5$ or $b = 0$. If $b = 0$, then $a$ is also 0 and there are no solutions. So we substitute 5 for $b$, yielding
\[4 = a + c.\]
We see that there are four possible values for $a$, yielding 15, 25, 35, and 45 as solutions. The requested sum is then $\boxed{120}$.

\pagebreak

\begin{problem}
    Five boys and six girls are to be seated in a row of eleven chairs so that they sit one at a time from one end to the other. The probability that there are no more boys than girls seated at any point during the process is $\frac{m}{n}$, where $m$ and $n$ are relatively prime positive integers. Evaluate $m + n$.
\end{problem}

\vspace{-\baselineskip}\rule{\textwidth}{0.4pt}

{\problemfont Solution Author} Henry McNamara

Think of this problem as counting the number of paths from $(0,0)$ to $(6,5)$ on a grid without going above the line $y = x$ and only taking steps from $(x,y)$ to $(x + 1,y)$ or $(x, y + 1)$. To get to the point $(6, 6)$, you must go through $(6, 5)$, so the number of valid arrangments is equal to the paths from $(0,0)$ to $(6,6)$, or the 6th Catalan Number. The total number of possible arrangements is ${11 \choose 5}$, so the probability is
\[\frac{C_{6}}{{11 \choose 5}} = \frac{2}{7},\]
yielding an answer of $2 + 7 = \boxed{009}$.

\pagebreak

\begin{problem}
    Point $P$ is situated inside regular hexagon $ABCDEF$ such that the feet from $P$ to $AB$, $BC$, $CD$, $DE$, $EF$, and $FA$ respectively are $G$, $H$, $I$, $J$, $K$, and $L$. Given that $PG = \frac{9}{2}$, $PI = 6$, and $PK = \frac{15}{2}$, the area of hexagon $GHIJKL$ can be written as $\frac{a\sqrt{b}}{c}$ for positive integers $a$, $b$, and $c$ where $a$ and $c$ are relatively prime and $b$ is not divisible by the square of any prime. Find $a + b + c$.
\end{problem}

\vspace{-\baselineskip}\rule{\textwidth}{0.4pt}

{\problemfont Solution Author} Henry McNamara

Connect $P$ to each vertex of $ABCDEF$ to get the three shaded areas below.

\begin{center}
    \begin{asy}
        size(8cm);
        pair A,B,C,D,E,F,G,H,I,J,K,L,P;
        A = (4sqrt(3),0);
        B = (2sqrt(3),6);
        C = (-2sqrt(3),6);
        D = (-4sqrt(3),0);
        E = (-2sqrt(3),-6);
        F = (2sqrt(3),-6);
        
        P = (sqrt(3)/2,3/2);

        G = foot(P,A,B);
        H = foot(P,B,C);
        I = foot(P,C,D);
        J = foot(P,D,E);
        K = foot(P,E,F);
        L = foot(P,F,A);

        fill(P -- A -- B -- cycle ^^ P -- C -- D -- cycle ^^ P -- E -- F -- cycle, palecyan);
        draw(A -- B -- C -- D -- E -- F -- cycle, RGB(63,159,255));
        draw(G -- H -- I -- J -- K -- L -- cycle, RGB(63,159,255));
        draw(H -- K ^^ G -- J ^^ I -- L, RGB(63,159,255));
        draw(P -- A ^^ P -- B ^^ P -- C ^^ P -- D ^^ P -- E ^^ P -- F, dashed+RGB(63,159,255));

        dot("$P$",P,dir(P));
        dot("$A$",A,dir(A));
        dot("$B$",B,dir(B));
        dot("$C$",C,dir(C));
        dot("$D$",D,dir(D));
        dot("$E$",E,dir(E));
        dot("$F$",F,dir(F));
        dot("$G$",G,dir(G));
        dot("$H$",H,dir(H));
        dot("$I$",I,dir(I));
        dot("$J$",J,dir(J));
        dot("$K$",K,dir(K));
        dot("$L$",L,dir(L));
    \end{asy}
\end{center}

\begin{claim*}
    For all $P$, the shaded area is half the area of $ABCDEF$.
\end{claim*}
\begin{proof}
    Imagine an equilateral triangle whose sides are extensions of $AB$, $CD$, and $EF$, and denote the height and side length of that triangle by $h$ and $s$, respectively. Notice that the area of the triangle can be expressed as
    \[\frac{1}{2}sh = \frac{1}{2}s(PG + PI + PK) \implies PG + PI + PK = h.\]
    The same argument holds for $PH + PJ + PL$ and their corresponding equilateral triangle congruent to the one above. So, we are done.
\end{proof}
Now, set the sidelength of $ABCDEF$ to equal $x$ and represent its area in two forms to yield
\[\left(6 + \frac{9}{2} + \frac{15}{2}\right)x = \frac{3\sqrt{3}x^{2}}{2}\]
\[x = 4\sqrt{3}.\]
It follows that the distance between parallel sides of $ABCDEF$ is 12, meaning
\[PH = \frac{9}{2}\]
\[PJ = \frac{15}{2}\]
\[PL = 6,\]
which makes the area (denoted by $N$),
\begin{align*}
    N &= \frac{1}{2}\sin\frac{\pi}{3}(PG \cdot PH + PH \cdot PI + PI \cdot PJ + PJ \cdot PK + PK \cdot PL + PL \cdot PG) \\
    &= \frac{\sqrt{3}}{4}\left(\frac{9}{2} \cdot \frac{9}{2} + \frac{9}{2} \cdot 6 + 6 \cdot \frac{15}{2} + \frac{15}{2} \cdot \frac{15}{2} + \frac{15}{2} \cdot 6 + 6 \cdot \frac{9}{2}\right) \\
    &= \frac{441\sqrt{3}}{8}.
\end{align*}
The desired sum is then $441 + 3 + 8 = \boxed{452}$.

\pagebreak

\begin{problem}
    The expected value of the square of the sum when rolling four standard six-sided dice can be written as $\frac{p}{q}$ for relatively prime positive integers $p$ and $q$. Find $p + q$.
\end{problem}

\vspace{-\baselineskip}\rule{\textwidth}{0.4pt}

{\problemfont Solution Author} Sarthak Jain

Let $X_i$ represent the outcome the $i$th dice roll. Using expected value, $\mathbb{E}[X_i] = \frac{7}{2}$, and variance, $\mathbb{V}[X_i] = \frac{35}{12}$, calculate
\begin{align*}
    \mathbb{E}[(X_1 + X_2 + X_3 + X_4)^2] &= \mathbb{E}[4X_i^2 + 12X_iX_j] \\ 
    &= 4\mathbb{E}[X_i^2] + 12\mathbb{E}[X_i]\mathbb{E}[X_j] \\
    &= 4 \left(\mathbb{V}[X_i] + \mathbb{E}[X_i]^2 \right) + 12\mathbb{E}[X_i]^2 \\
    &= 16\left(\frac{7}{2}\right)^2 + 4\left(\frac{35}{12}\right) \\
    &= 196 + \frac{35}{3} \\
    &= \frac{623}{3}.
\end{align*}
Finally, $\frac{623}{3} \implies 623 + 3 = \boxed{626}$.

\pagebreak

{\problemfont Solution Author} Henry McNamara

We solve this problem using expected value. From direct calculation, the expected value of the $i$th roll is $\mathbb{E}[X_{i}] = \frac{91}{6}$ and that of the square of the $i$th roll is $\mathbb{E}[X_{i}^{2}] = \frac{91}{6}$. Now, using properties of expected value, we find
\begin{align*}
    \mathbb{E}[(X_{1} + X_{2} + X_{3} + X_{4})^{2}] &= \mathbb{E}[16X_{i}^{2}] \\
    &= \mathbb{E}[4X_{i}^{2} + 12X_{i}X_{j}] \\
    &= 4\mathbb{E}[{X_{i}^{2}}] + 12\mathbb{E}[X_{i}]\mathbb{E}[X_{j}] \\
    &= 4 \cdot \frac{91}{6} + 12 \cdot \left(\frac{7}{2}\right)^{2} \\
    &= \frac{623}{3}
\end{align*}
So, the final answer is $623 + 3 = \boxed{626}$.

\pagebreak

\begin{problem}
    The value of $x$ which satisfies
    \[1 + \log_{x}(\floor{x}) = 2\log_{x}(\sqrt{3}\{x\})\]
    can be written in the form $\frac{a + \sqrt{b}}{c}$, where $a$, $b$, and $c$ are positive integers and $b$ is not divisible by the square of any prime. Find $a + b + c$.
    
    \emph{Note: $\floor{x}$ denotes the greatest integer less than or equal to $x$ and $\{x\}$ denotes the fractional part of $x$.}    
\end{problem}

\vspace{-\baselineskip}\rule{\textwidth}{0.4pt}

{\problemfont Solution Author} Henry McNamara

Begin by eliminating the logarithms to yield
\[x\floor{x} = 3\{x\}^{2}.\]
For the sake of simplicity, we let $\floor{x} = n$ and $\{x\} = p$. So,
\[(n + p)(n) = 3p^{2}\]
\[n^{2} + np = 3p^{2}\]
\[3p^{2} - np - n^{2} = 0.\]
Now, express $p$ in terms of $n$, as
\begin{align*}
    p &= \frac{n \pm \sqrt{n^{2} + 12n^{2}}}{6} \\
    &= \frac{n \pm \sqrt{13}n}{6}.
\end{align*}
Since $p$ is non-negative,
\[p = \frac{1 + \sqrt{13}}{6}n.\]
Because $3p^{2} < 3$, $n^{2} + np < 3$ as well, meaning $n < \sqrt{3}$. If $n = 0$, then so does $p$ and the logarithms are undefined. So, $n = 1$ and $p = \frac{1 + \sqrt{13}}{6}$, making $x = \frac{7 + \sqrt{13}}{6}$. It follows that $a + b + c = 7 + 13 + 6 = \boxed{026}$.

\pagebreak

\begin{problem}
    Triangle $ABC$ has side lengths $AB = 2$, $AC = 6.5$, and $BC = 7.5$. Points $M$ and $N$ lie on $AB$ and $AC$ respectively so that $MC$ and $NB$ intersect at point $O$. If triangles $MBO$ and $NCO$ both have area 1, the product of all possible areas of triangle $AMN$ can be written as $\frac{p}{q}$ for relatively prime positive integers $p$ and $q$. Find $p + q$. 
\end{problem}

\vspace{-\baselineskip}\rule{\textwidth}{0.4pt}

{\problemfont Solution Author} Sarthak Jain

\begin{center}
    \begin{asy}
        size(12cm);
        
        real n = 87/20;
        pair A, B, C, M, N, O, f1, f2, f3;
        path h1, h2, h3;
        A = (6/5, 8/5);
        B = origin;
        C = (15/2, 0);
        M = (3/5, 4/5);
        N = (n, 8 * (15 - 2n) / 63);
        O = intersectionpoint(M -- C, N -- B);
        f1 = foot(A,B,C);
        f2 = foot((M + N)/2,B,C);
        f3 = foot(O,B,C);
        h1 = A -- f1;
        h2 = (M + N)/2 -- f2;
        h3 = O -- f3;
    
        fill(A -- M -- N -- cycle, palecyan);
        draw(A -- B -- C -- cycle, RGB(63,159,255));
        draw(M -- N, RGB(63,159,255));
        draw(M -- C, RGB(63,159,255));
        draw(N -- B, RGB(63,159,255));
        draw(h1 ^^ h2 ^^ h3, RGB(63,159,255));
    
        dot("$O$",O,E);
        dot("$A$",A,NE);
        dot("$B$",B,SW);
        dot("$C$",C,SE);
        dot("$M$",M,W);
        dot("$N$",N,NE);
        label("$h_{1}$",h1,E);
        label("$h_{2}$",h2,E);
        label("$h_{3}$",h3,E);
    \end{asy}
\end{center}

From the properties of trapezoids, we deduce that since $[BCM] = [BCN]$, we have $MN \parallel BC$. Therefore $\triangle AMN \sim \triangle ABC$. Assume the ratio of their side lengths is $p : 1$. Likewise notice that $\triangle MNO \sim \triangle CBO$ with the same similarity ratio. Let $h_1$ be the height from $A$ to $BC$, $h_2$ be the distance between $MN$ and $BC$, and $h_3$ be the height from $O$ to $BC$. It follows that
\[h_2 = h_1(1-p)\]
\[h_3 = h_2 \frac{1}{1+p} = h_1 \frac{1-p}{1+p}.\]
Notice that $[ABC] = \frac{1}{2}BC \cdot h_1 = 6$ (from Heron's Formula). We can solve for $p$ by solving for the area of $\triangle BNC$ in multiple ways. Namely,
\[[BNC] = \frac{1}{2}(BC \cdot h_2) = \frac{1}{2}(BC \cdot h_3) + 1\]
\[\frac{1}{2} \cdot BC \cdot h_1(1-p) = \frac{1}{2} \cdot BC \cdot h_1 \frac{1-p}{1+p} + 1\]
\[6(1-p) = 6 \frac{1-p}{1+p} + 1\]
\[6(1-p^2) = 7 - 5p\]
\[(3p-1)(2p-1) = 0.\]
Now we have, $p = \frac{1}{2}$ or $\frac{1}{3}$. Since $[ABC] \cdot p^2 = [AMN]$, the sum of all possible $[AMN]$ is $\frac{1}{4} \cdot 6 + \frac{1}{9} \cdot 6 = \frac{13}{6}$. So our answer is $\boxed{019}$.

\pagebreak

\begin{problem}
    Adam and Bettie are playing a game. They take turns generating a random number between 0 and 127 inclusive. The numbers they generate are scored as follows:
    \begin{itemize}
        \item If the number is zero, it receives no points.
        \item If the number is odd, it receives one more point than the number one less than it.
        \item If the number is even, it receives the same score as the number with half its value.
    \end{itemize}
    if Adam and Bettie both generate one number, the probability that they receive the same score is $\frac{p}{q}$ for relatively prime positive integers $p$ and $q$. Find $p$.
\end{problem}

\vspace{-\baselineskip}\rule{\textwidth}{0.4pt}

{\problemfont Solution Author} Henry McNamara

Observe that the score of $n$ is simply the number of $1$s in the binary representation of $n$. To calculate the probability, consider a sequence of seven $1$s and $0$s. There are $2^{7}$ possible sequences. The probability that a given sequence contains $k$ $1$s is
\[\frac{{7 \choose k}}{2^7}.\]
So, the desired probability is
\[P = \sum_{k = 0}^{7}\left(\frac{{7 \choose k}}{2^7}\right)^{2}.\]
We use the combinatorial identity
\[\sum_{k = 0}^{n} {n \choose k}^{2} = {2n \choose n}\]
to obtain
\[P = \frac{{14 \choose 7}}{2^{14}} = \frac{429}{2^{11}}.\]
So, $p = \boxed{429}$.

\pagebreak

\begin{problem}
    The function $y = x^{2}$ is graphed in the $xy$-plane. A line from every point on the parabola is drawn to the point $(0, -10, a)$ in three-dimensional space. The locus of points where the lines intersect the $xz$-plane forms a closed path with area $\pi$. Given that $a = \frac{p\sqrt{q}}{r}$ for positive integers $p$, $q$, and $r$ where $p$ and $r$ are relatively prime and $q$ is not divisible by the square of any prime, find $p + q + r$.
\end{problem}

\vspace{-\baselineskip}\rule{\textwidth}{0.4pt}

{\problemfont Solution Author} Henry McNamara

There are two key observations for this problem:
\begin{enumerate}
    \item The closed path is an ellipse.
    \item The points at infinity (or on the horizon line) pass through the $xz$-plane at $z = a$.
\end{enumerate}
Thus, one axis of the ellipse has length $a$. To find the other, we look for the points that pass through the line $z = \frac{a}{2}$ in the $xz$-plane. By symmetry, those are the points $(\sqrt{10},10)$ and $(-\sqrt{10},10)$. However, the points where the two lines cross the $xz$-plane are half as far apart as their corresponding points in the $xy$-plane, making the second axis have length $\sqrt{10}$. Now, we solve for $a$:
\[\frac{\sqrt{10}}{2} \cdot \frac{a}{2} \cdot \pi = \pi\]
\[a = \frac{2\sqrt{10}}{5}.\]
So, $p + q + r = \boxed{017}$.

\pagebreak

{\problemfont Solution Author} vanstraelen (AoPS)

If you do not realize that the closed path is an ellipse, the problem can be solved as follows. Let $B = (0, -10, a)$ and each point on the parabola be some point $A = (\lambda, \lambda^{2}, 0)$. Using parametric equations, we get
\[AB = \begin{cases}
    x = \lambda t \\
    y = -10 + (\lambda^{2} + 10)t \\
    z = a - at
\end{cases}.\]
So,
\[\frac{x}{\lambda} = \frac{y + 10}{\lambda^{2} + 10} = \frac{z - a}{-a}.\]
When $AB$ intersects the $xz$-plane, $y = 0$, which makes
\[x = \frac{10\lambda}{\lambda^{2} + 10} \;\; \text{and} \;\; z = \frac{a\lambda^{2}}{\lambda^{2} + 10}.\]
From here, we eliminate $\lambda$. After some lengthy simplification, we get the same ellipse as above, yielding an answer of $\boxed{017}$.

\pagebreak

\begin{problem}
    Find the sum of all positive integers $n$ such that the values of $\floor{\frac{n}{3}}$, $\floor{\frac{n}{4}}$, and $\floor{\frac{n}{5}}$ form an arithmetic sequence.
\end{problem}

\vspace{-\baselineskip}\rule{\textwidth}{0.4pt}

{\problemfont Solution Author} Henry McNamara

The problem equivalently states
\[2\floor{\frac{n}{4}} = \floor{\frac{n}{3}} + \floor{\frac{n}{5}}.\]
Rewrite $n$ as $n = 15k + a$ where $a \leq 14$, making the above
\begin{align*}
    2\floor{\frac{15k + a}{4}} &= \floor{\frac{15k + a}{3}} + \floor{\frac{15k + a}{5}} \\
    &= 8k + \floor{\frac{a}{3}} + \floor{\frac{a}{5}}.
\end{align*}
Observe that the right side of the above equation must be even. So, when $a \in \{3, 4, 6, 7, 8, 10, 11\}$, there are no solutions. To find valid $n$, notice it holds that
\[8k + \floor{\frac{a}{3}} + \floor{\frac{a}{5}} \leq \frac{15k + a}{2} < 8k + \floor{\frac{a}{3}} + \floor{\frac{a}{5}} + 2\]
First, we analyze the right side:
\[\frac{15k + a}{2} < 8k + \floor{\frac{a}{3}} + \floor{\frac{a}{5}} + 2\]
\[a - 2\floor{\frac{a}{3}} - 2\floor{\frac{a}{5}} - 4 < k.\]
For all valid $a$ and $k$, the inequality holds. So, we only must look at the left side:
\[8k + \floor{\frac{a}{3}} + \floor{\frac{a}{5}} \leq \frac{15k + a}{2}\]
\[k \leq a - 2\floor{\frac{a}{3}} - 2\floor{\frac{a}{5}}.\]
Proceed with casework on $a$:
\[a = 0 \implies k = 0 \implies n \in \{\emptyset\}\]
\[a = 1 \implies k \in \{0, 1\} \implies n \in \{1, 16\}\]
\[a = 2 \implies k \in \{0, 1, 2\} \implies n \in \{2, 17, 32\}\]
\[a = 5 \implies k \in \{0, 1\} \implies n \in \{5, 20\}\]
\[a = 9 \implies k \in \{0, 1\} \implies n \in \{9, 24\}\]
\[a = 12 \implies k = 0 \implies n = 12\]
\[a = 13 \implies k \in \{0, 1\} \implies n \in \{13, 28\}\]
\[a = 14 \implies k \in \{0, 1, 2\} \implies n \in \{14, 29, 44\}.\]
Adding the solutions yields an answer of $\boxed{266}$.

\pagebreak

\begin{problem}
    One face of a tetrahedron has sides of length 3, 4, and 5. The tetrahedron's volume is 24 and surface area is $n$. When $n$ is minimized, it can be expressed in the form $n = a\sqrt{b} + c$, where $a$, $b$, and $c$ are positive integers and $b$ is not divisible by the square of any prime. Evaluate $a + b + c$.
\end{problem}

\vspace{-\baselineskip}\rule{\textwidth}{0.4pt}

{\problemfont Solution Author} Sarthak Jain and P\_Groudon (AoPS)

Let the three vertices of the defined face be $A$, $B$, and $C$ so that $AB = 3$, $BC = 4$, and $AC = 5$. Let $D$ be the final vertex of the tetrahedron and $E$ the foot from $D$ to the plane determined by triangle $ABC$. From the volume condition, $DE = 12$. Define the distances from $E$ to $AB$, $BC$, and $AC$ as $x$, $y$, and $z$, respectively. Finally, let the altitudes from $D$ in triangles $DAB$, $DBC$, and $DAC$ have length $h_{1}$, $h_{2}$, and $h_{3}$, respectively. We can then represent the surface area of the tetrahedron as
\begin{align*}
    K &= [ABC] + [ABD] + [ACD] + [BCD] \\
    &= 6 + \frac{1}{2}\left(3h_{1} + 4h_{2} + 5h_{3}\right) \\
    &= 6 + \frac{1}{2}\left(3\sqrt{x^{2} + 144} + 4\sqrt{y^{2} + 144} + 5\sqrt{z^{2} + 144}\right)
\end{align*}

\begin{claim*}
    The surface area is minimized when $x = y = z = 1$.
\end{claim*}
\begin{proof}
    We must only consider the expression
    \[3\sqrt{x^{2} + 144} + 4\sqrt{y^{2} + 144} + 5\sqrt{z^{2} + 144},\]
    which can be rewritten as
    \[\sqrt{(3x)^{2} + (3 \cdot 12)^{2}} + \sqrt{(4y)^{2} + (4 \cdot 12)^{2}} + \sqrt{(5z)^{2} + (5 \cdot 12)^{2}}.\]
    Notice that the above is the length of the path
    \[(0, 0) \rightarrow (3x, 36) \rightarrow (3x + 4y, 84) \rightarrow (3x + 4y + 5z, 144),\]
    which is minimized when the points are all colinear. However, by representing the area of $ABC$ in terms of $x$, $y$, and $z$, we have
    \[3x + 4y + 5z = 12.\]
    So, the colinearity occurs when $x = y = z = 1$.
\end{proof}
This makes the area
\[K = 6 + 6\sqrt{145} \implies a + b + c = \boxed{157}.\]

\pagebreak

\begin{problem}
    What is the least positive integer $n$ such that $25^{n} + 16^{n}$ leaves a remainder of 1 when divided by 121?
\end{problem}

\vspace{-\baselineskip}\rule{\textwidth}{0.4pt}

{\problemfont Solution Author} Henry McNamara

First, we analyze modulo $11$:
\[25^{n} + 16^{n} \equiv 3^{n} + 5^{n} \pmod{11}.\]
Looking at powers of 3 and 5 yields,
\begin{center}
    \begin{tabular}{l | r r r r r}
        n & 1 & 2 & 3 & 4 & 5 \\
        \hline
        $3^{n} \pmod{11}$ & 3 & 9 & 5 & 4 & 1 \\
        $5^{n} \pmod{11}$ & 5 & 3 & 4 & 9 & 1 \\
    \end{tabular}
\end{center}    
a pattern that cycles with length 5. Notice that $3^{n} + 5^{n} \equiv 1 \pmod{11}$ implies $n = 5k + 2$ for some integer $k$. So,
\[25^{5k + 2} + 16^{5k + 2} \equiv 1 \pmod{121}.\]
Now, rewriting the above allows us to use binomial expansion:
\[(22 + 3)^{5k + 2} + (11 + 5)^{5k + 2} \equiv 1 \pmod{121}\]
\[(22 \cdot 3^{5k + 1} + 11 \cdot 5^{5k + 1})(5k + 2) + 3^{5k + 2} + 5^{5k + 2} \equiv 1 \pmod{121}.\]
Observe that $3^{5k + 1} = 11a + 3$ and $5^{5k + 1} = 11b + 5$, for some integers $a$ and $b$. This yields
\[(22 \cdot (11a + 3) + 11 \cdot (11b + 5))(5k + 2) + 3^{5k + 2} + 5^{5k + 2} \equiv 1 \pmod{121},\]
which simplifies to
\[3^{5k + 2} + 5^{5k + 2} \equiv 1 \pmod{121}.\]
Notice that $3^{5} \equiv 1 \pmod{121}$. So,
\[9 + 5^{5k + 2} \equiv 1 \pmod{121}\]
\[5^{5k + 2} \equiv -8 \pmod{121}\]
\[25 \cdot 5^{5k} \equiv -8 \pmod{121}.\]
Since 92 is the modular inverse of 25 with respect to 121,
\[5^{5k} \equiv -10 \pmod{121}\]
Simplify again by noticing $5^{5} \equiv 100 \pmod{121}$ and using binomial expansion to yield
\[99k \equiv -11 \pmod{121}\]
\[9k \equiv -1 \pmod{11}\]
\[k \equiv 6 \pmod{11},\]
which makes the minimum value of $n$ equal to $6 \cdot 5 + 2 = \boxed{032}$.

\pagebreak

\begin{problem}
    For a sequence $s = (s_{1}, s_{2}, \dots, s_{n})$, define
    \[F(s) = \sum_{i = 1}^{n - 1} (-1)^{i + 1}(s_{i} - s_{i + 1})^{2}.\]

    Consider the sequence $S = (2^{1}, 2^{2}, \dots, 2^{1000})$. Let $R$ be the sum of all 
    $F(m)$ for all non-empty \emph{subsequences} $m$ of $S$. Find the remainder when $R$ is divided by 1000.

    \emph{Note: A subsequence is a sequence that can be obtained from another sequence by deleting some non-negative number of values without changing the order.}
\end{problem}

\vspace{-\baselineskip}\rule{\textwidth}{0.4pt}

{\problemfont Solution Author} Sarthak Jain

We can rewrite the total sum over all subsequences of $S$,  
\[R = \sum_{m\subseteq S}F(m) = \sum_{m\subseteq S}\sum_{i=1}^{\abs{m} - 1}(-1)^{i+1}(2^{i} - 2^{i + 1})^2 = \sum_{1\leq i < j \leq 1000} (p_{i,j}-n_{i,j})(2^{i} - 2^{j})^2\]
where $p_{i,j}$ and $n_{i,j}$ represent the number of positive and negative occurrences, respectively, of the element pair $(2^{i},2^{j})$ when they are adjascent in a subsequence.

\begin{claim*}
    $p_{i,j}-n_{i,j} = 0$ for all $i \neq 1$.
\end{claim*}
\begin{proof}
    For any given $(2^{i},2^{j})$ as an adjascent pair of elements in some subsequence, $m$, the sign of their difference is determined by the index of $2^{i}$, which we call $k \leq i$. We add one to $p_{i,j}$ if $k$ is odd. Otherwise, we add to $n_{i,j}$. Since $k$ is the index of the first element in the pair, there are $k-1$ elements before it with indices in $\{1, 2, \dots, i - 1\}$. From this, there are $2^{i - 1}$ combinations of elements that can be included in any given subsequence. Thus,
    \[p_{i,j} = {i - 1 \choose 0} + {i - 1 \choose 2} + \cdots\]
    and
    \[n_{i,j} = {i - 1 \choose 1} + {i - 1 \choose 3} + \cdots.\]
    Consider the polynomial $P(x) = (x - 1)^{i - 1}$ with coefficients $a_{1}, a_{2}, \dots, a_{i - 1}$. It holds that
    \begin{align*}
        p_{i,j} - n_{i,j} &= \sum_{l = 1}^{i - 1} a_{l} \\
        &= P(1) \\
        &= 0
    \end{align*}
    for all $i \geq 2$. Otherwise, when $i = 1$, the above equality does not hold.
\end{proof}

When $i = 1$, $n_{i,j} = 0$ since there are no elements before the first. Also, $p_{1,j}$ is simply the number of subsequences consisting of the last $1000-j$ terms, which is $2^{1000-j}$. Now, we simplify $R$ to
\begin{align*}
    R &= \sum_{1 \leq i < j \leq 1000} (p_{i,j} - n_{i,j})(2^{i} - 2^{j})^{2} \\
    &= \sum_{j = 2}^{1000} (p_{1,j} - n_{1,j})(2 - 2^{j})^{2} \\
    &= \sum_{j = 2}^{1000} 2^{1000 - j}(2 - 2^{j})^{2} \\
    &= \sum_{j = 2}^{1000} 2^{1000 - j}(2^{2j} - 2^{j + 2} + 4) \\
    &= 2^{1000}\sum_{j = 2}^{1000} (2^{j} - 4 + 2^{2 - j}) \\
    &= 2^{1000} (2^{1001} - 2 - 4 \cdot 999 - 2^{-998}).
\end{align*}
Proceed with analyzing $R \pmod{1000}$:
\begin{align*}
    R &\equiv 2^{1000} (2^{1001} + 2 - 2^{-998}) \pmod{1000} \\
    &\equiv 2^{2001} + 2^{1001} - 4 \pmod{1000} \\
\end{align*}
Notice $R \equiv 4 \pmod{8}$ and, from Euler's Totient Theorem, $2^{100} \equiv 1 \pmod{125}$, which makes $R \equiv 0 \pmod{125}$. So, by the Chinese Remainder Theorem, $R \equiv \boxed{500} \pmod{1000}$.

\pagebreak

\begin{problem}
    The Fibonacci Sequence is defined as follows: $F_{0} = 0$, $F_{1} = 1$, and $F_{n} = F_{n - 1} + F_{n - 2}$ for integers $n \geq 2$. The sum
    \[S = \sum_{n = 0}^{\infty} \frac{F_{n}^{2}}{9^{n}}\]
    can be written as $\frac{m}{n}$ where $m$ and $n$ are relatively prime positive integers. Find $m + n$.
\end{problem}

\vspace{-\baselineskip}\rule{\textwidth}{0.4pt}

{\problemfont Solution Author} Sarthak Jain

Using $F_n = F_{n-1} + F_{n-2}$,
\begin{align*}
    S &= \sum_{n=0}^{\infty} \frac{(F_n)^2}{9^n} \\
    &= 0 + \frac19 + \sum_{n=2}^{\infty} \frac{(F_n)^2}{9^n} \\
    &= \frac19 + \sum_{n=2}^{\infty} \frac{(F_{n-1}+F_{n-2})^2}{9^n} \\
    &= \frac19 + \sum_{n=2}^{\infty} \frac{(F_{n-1})^2}{9^n} + \sum_{n=2}^{\infty} \frac{(F_{n-2})^2}{9^n} + 2\sum_{n=2}^{\infty} \frac{(F_{n-1}F_{n-2})}{9^n} \\
    &=\frac19 + \frac19S + \frac{1}{9^2}S + 2A \\
    &=\frac19 + \frac{10}{81}S + \frac{1}{36}S \\
    &= \frac19 + \frac{49}{324}S \\
    &= \frac19 \cdot \frac{324}{324-49} \\
    &= \frac{36}{275}
\end{align*}
The value of $A$ used above is as follows:
\begin{align*}
    A &= \sum_{n=2}^{\infty}\frac{F_{n-1}F_{n-2}}{9^n} \\
    &= 0 + \sum_{n=3}^{\infty}\frac{(F_{n-2} + F_{n-3})F_{n-2}}{9^n} \\
    &= \frac1{81}S + \frac19A \\
    &= \frac1{72}S
\end{align*}
Finally, $S = \frac{m}{n} = \frac{36}{275} \implies m+n = \boxed{311}$. 

\pagebreak

{\problemfont Solution Author} Henry McNamara

We use Binet's formula to represent each term as
\begin{align*}
    F_{n}^{2} &= \frac{(\phi^{n} - (1 - \phi)^{n})^{2}}{5} \\
    &= \frac{\phi^{2n} - 2\phi^{n}(1 - \phi)^{n} + (1 - \phi)^{2n}}{5} \\
    &= \frac{\phi^{2n} - 2(\phi - \phi^{2})^{n} + (1 - \phi)^{2n}}{5},
\end{align*}
which makes the sum equal to
\begin{align*}
    S &= \frac{1}{5}\sum_{n = 0}^{\infty} \frac{\phi^{2n} - 2(\phi - \phi^{2})^{n} + (1 - \phi)^{2n}}{9^{n}} \\
    &= \frac{1}{5}\sum_{n = 0}^{\infty} \frac{\phi^{2n}}{9^{n}} - \frac{2}{5}\sum_{n = 0}^{\infty} \frac{(\phi - \phi^{2})^{n}}{9^{n}} + \frac{1}{5}\sum_{n = 0}^{\infty} \frac{(1 - \phi)^{2n}}{9^{n}} \\
    &= \frac{1}{5} \cdot \frac{1}{1 - \frac{\phi^{2}}{9}} - \frac{2}{5} \cdot \frac{1}{1 - \frac{\phi - \phi^{2}}{9}} + \frac{1}{5} \cdot \frac{1}{1 - \frac{(1 - \phi)^{2}}{9}} \\
    &= \frac{1}{5} \cdot \frac{1}{1 - \frac{3 + \sqrt{5}}{18}} - \frac{2}{5} \cdot \frac{1}{1 + \frac{1}{9}} + \frac{1}{5} \cdot \frac{1}{1 - \frac{3 - \sqrt{5}}{18}} \\
    &= \frac{1}{5} \cdot \frac{1}{\frac{15 - \sqrt{5}}{18}} - \frac{2}{5} \cdot \frac{9}{10} + \frac{1}{5} \cdot \frac{1}{\frac{15 + \sqrt{5}}{18}} \\
    &= \frac{\frac{15 + \sqrt{5}}{18}}{\frac{275}{81}} - \frac{9}{25} + \frac{\frac{15 - \sqrt{5}}{18}}{\frac{275}{81}} \\
    &= \frac{30 \cdot 81}{275 \cdot 18} - \frac{9}{25} \\
    &= \frac{54}{110} - \frac{9}{25} \\
    &= \frac{36}{275}.
\end{align*}
So, $m + n = \boxed{311}$.

\pagebreak

\begin{problem}
    The roots of the polynomial $P(x) = x^{4} - 10x^{3} + 35x^{2} - 51x + 26$ form the side lengths of a quadrilateral. The maximum area of such a quadrilateral is of the form $\sqrt{n}$ for some positive integer $n$. Find $n$.
\end{problem}

\vspace{-\baselineskip}\rule{\textwidth}{0.4pt}

{\problemfont Solution Author} Sarthak Jain

From Bretschneider's Formula, 
\[[ABCD] = \sqrt{(s-a)(s-b)(s-c)(s-d) - abcd \cdot \cos^2{\left(\frac{A+C}{2}\right)}},\]
where $a$, $b$, $c$, and $d$ are the side lengths and $s$ is the semiperimeter. It is clear that the area is maximized when $A+C = 180^\circ$. Thus,
\[\max{ABCD} = \sqrt{(s-a)(s-b)(s-c)(s-d)} = \sqrt{P(s)}.\]
From Vieta's Formulas, $s = \frac{a+b+c+d}{2} = \frac{10}{2} = 5$. So, $\sqrt{P(s)} = \sqrt{P(5)} = \sqrt{21}$, which yields an answer of $\boxed{21}$.

\pagebreak

\begin{problem}
    Triangle $ABC$ satisfies $\tan{A} \cdot \tan{B} = 3$ and $AB = 5$. Let $G$ and $O$ be the centroid and circumcenter of $ABC$ respectively. The maximum possible area of triangle $CGO$ can be written as $\frac{a\sqrt{b}}{c}$ for positive integers $a$, 
    $b$, and $c$ with $a$ and $c$ relatively prime and $b$ not divisible by the square of any prime. Find $a + b + c$.
\end{problem}

\vspace{-\baselineskip}\rule{\textwidth}{0.4pt}

{\problemfont Solution Author} Henry McNamara

Define $H$ as the orthocenter, $K$ the foot from $C$ to $AB$, and $M$ the midpoint of $AB$.

\begin{center}
    \begin{asy}
        size(8cm);

        pair A,B,C,p1,p2,H,O,G,K,L,M,P,Q;

        A = origin;
        B = (5,0);
        p1 = 10*(1,2*sqrt(3));
        p2 = B + 10*(-1,sqrt(3)/2);

        path ap1 = A -- p1;
        path bp2 = B -- p2;

        C = intersectionpoint(ap1,bp2);

        K = foot(C,A,B);
        L = 0.5*(A+B);
        M = foot(B,A,C);
        H = intersectionpoint(C -- K, B -- M);
        O = circumcenter(A,B,C);
        G = centroid(A,B,C);
        P = (A + C)/2;
        Q = (B + C)/2;

        fill(C -- G -- O -- cycle, palecyan);

        draw(A -- B -- C -- cycle, RGB(63,159,255));
        draw(H -- O, RGB(63,159,255));
        draw(C -- K, RGB(63,159,255));
        draw(C -- L, RGB(63,159,255));
        draw(C -- O, RGB(63,159,255));
        draw(O -- L, RGB(63,159,255));
        // draw(O -- P, RGB(63,159,255));
        // draw(O -- Q, RGB(63,159,255));

        dot("$A$",A,SW);
        dot("$B$",B,SE);
        dot("$C$",C,N);
        dot("$O$",O,E);
        dot("$H$",H,W);
        dot("$G$",G,NE);
        dot("$K$",K,S);
        dot("$M$",L,S);
        // dot("$P$",P,NW);
        // dot("$Q$",Q,NE);

    \end{asy}
\end{center}

\begin{claim*}
    The Euler line of triangle $ABC$ is parallel to $AB$.
\end{claim*}
\begin{proof}
    Let the radius of the circumcircle of $ABC$ be $R$. We now wish to show that $OM = HK$. First, we find $OM$. Notice that
    \[\angle OBM = 90\dg - \angle C \implies OM = R(\sin A \sin B - \cos A \cos B).\]
    Next, if the Euler line is parallel to $AB$, then $HK = \frac{1}{3}CK$. So, we find $CK$. Start with
    \[\angle CAO = 90\dg - \angle B \implies AC = 2R \sin B.\]
    It follows that
    \[\angle ACK = 90\dg - \angle A \implies CK = 2R \sin B \sin A \implies HK = \frac{2}{3}R \sin B \sin A.\]
    Now, equate $HK$ and $OM$:
    \[\frac{2}{3} \sin B \sin A = \sin A \sin B - \cos A \cos B\]
    \[\cos A \cos B = \frac{1}{3}\sin A \sin B\]
    \[\tan A \tan B = 3.\]
    So, we are done.
\end{proof}

From the property of the Euler Line stating
\[HG : GO = 2 : 1,\]
and the centroid trisecting the medians,
\[[CGO] = \frac{1}{9}KM \cdot KC.\]
From the tangent condition, notice
\[CK^{2} = 3AK \cdot BK.\]
We let $AK = x$, making
\[CK^{2} = 3x(5 - x)\]
and
\[MK^{2} = \left(\frac{5}{2} - x\right)^{2}.\]
It follows that
\[3MK^{2} + CK^{2} = \frac{75}{4}.\]
By AM-GM, the area is maximized when
\[\frac{3MK^{2} + CK^{2}}{2} = \sqrt{3}MK \cdot CK,\]
or when
\[\frac{1}{9}MK \cdot CK = \frac{25\sqrt{3}}{72},\]
which implies a sum of $\boxed{100}$.

\end{document}