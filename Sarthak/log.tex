\documentclass[11pt]{scrartcl}

\usepackage[sexy]{evan}
\usepackage[parfill]{parskip}
\usepackage{amsmath}

\title{Sarthak Log Floor Problem}

\begin{document}

The value of $x$ which satisfies
\[1 + \log_{x}(\floor{x}) = 2\log_{x}(\sqrt{3}\{x\})\]
can be written in the form $\frac{a + \sqrt{b}}{c}$, where $a$, $b$, and $c$ are relatively prime integers, and $b$ is not divisible by the square of any prime. Find $a + b + c$.

Here, $\floor{x}$ denotes the greatest integer less than or equal to $x$ and $\{x\}$ denotes the fractional part of $x$.

\end{document}